%
An extension of the concept of dispersion is the \enquote{p-dispersion} problem, defined for instance by \cto[section 3]{erkut1990discrmaximin}. The idea is to find a subset of points $\mathcal U \subset \fnP_i$ with maximal smallest pair-wise distance $d(\cdot,\cdot)$,
\begin{align}\label{eq:pdisp}
  \pdisp(\fnP_i, p) = \max_{\stackrel{\mathcal U \subseteq \fnP_i}{\abs{\mathcal U} = p}} \quad\min_{\bm p_i, \bm p_j \in \mathcal U, i \neq j} d(\bm p_i, \bm p_j).
\end{align}
%\paragraph{Interpreting p-dispersion}

For $p=2$, $\pdisp(\cdot,2)$ equals the greatest distance between two points. Likewise, $\pdisp(\fnP_i, \abs{\fnP_i})$ computes the smallest of such distances. More interestingly, $\pdisp(\fnP_i,3)$ considers all triangles with vertices matching points of $\fnP_i$, folds each triangle to its smallest side, yielding a list line segments out of which the greatest is chosen. In contrast to $\disp(\fnP_i, \mathcal R, \mu)$, no emptiness condition is applied to these triangles. For $p>3$ this situation is analogue.

\paragraph{Computational complexity}

A reduction to the CLIQUE problem was provided by \ct{erkut1990discrmaximin}. The CLIQUE problem itself is \ccNPcomplete, which was shown by \ct{cook1971sat, karp1972cclass}. Nevertheless approximating this problem appears to demand caution, since \ct{feige1991clique} have shown that the existance of a good approximation implies $\mathcal{NP} = \mathcal{P}$ being considered invalid but for which a proof is still pending. Therefore, approximating the p-dispersion problem appears to require care.
