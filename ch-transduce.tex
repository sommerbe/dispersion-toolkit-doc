%
An abundance of point set generating softwares, each streaming a custom format, entails a necessity to transduce these formats into the internally employed format of this dispersion toolkit. This compatibility layer emerges out of the conception to modularise software development to reduce the duplication of source code.

This chapter covers the transformation of file formats storing point sets and sequences thereof. It begins with both reading and writing matrices. Afterwards, it explains how to convert point sets generated by the UTK framework into the format used by this toolkit.

\begin{figure}[tb]
  \begin{verbatim}# comment: dptk point set sequence file format
#d 0.0e+00 0.0e+00 1.0e+00 1.0e+00
0.0000000000000000e+00 0.0000000000000000e+00
7.6923076923076927e-02 3.8461538461538464e-01
1.5384615384615385e-01 7.6923076923076927e-01
#eos\end{verbatim}%2.3076923076923078e-01 1.5384615384615397e-01
  \vskip 1em
  $\xrightarrow{\text{transduce with}} \codef{write-matrix --mathematica}$
  \vskip 1em
  \begin{verbatim}{{0.0e+00,0.0e+00},{7.6923076923076927e-02,3.8461538461538464
e-01},{1.5384615384615385e-01,7.6923076923076927e-01}}\end{verbatim}
\end{figure}

\clearpage
