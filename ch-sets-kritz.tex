
Achieving smaller dispersion, \ct{kritzinger2020} modified this Fibonacci lattice with
\begin{align*}
  \mathcal Z_n &\coloneqq [0, n-1]_{\mathbb N_0}, \\
  \pi(j) &\coloneqq j F_{m-2}\ \text{mod}\ F_m, \\
  s(i) &\coloneqq \begin{cases}
    \mleft(\sqrt{5}+1\mright)2^{-1} & \text{if}\ \pi(i) < \pi(i+1), \\
    1 & \text{else},
  \end{cases} \\
  x_j &\coloneqq \sum_{i\,\in\,\mathcal Z_j} s(i), \\
  \fnPi{K}(m) &\coloneqq \mleft\{ \bm s_k \coloneqq (x_k, x_{\pi(k)}) x_{n-1}^{-1}\ .\ \forall k \in \mathcal Z_n,\ n = F_m \mright\}.
\end{align*}
This construction yields a point set with smaller dispersion than the standard Fibonacci set, when using empty rectangles as regions with the area measure. In particular, these regions show a tendency to be larger in size closer to the domain boundary, while leading to more greatest empty regions within the entire domain.


\paragraph{Procedure}

\begin{synopsis}
  \subsynopsis{set-kritzinger}{(--fibonacci-index|--m)=$m$\newline 
    [--o=FILE]\newline
    [--delimiter=CHARACTER] [--silent]}
  \subsynopsis{set-kritzinger}{(--fibonacci-index|--m)=$m$\newline
    --compute-fibonacci-number|--cardinality [--no-pointset]\newline
    [--o=FILE]\newline
    [--delimiter=CHARACTER] [--silent]}
\end{synopsis}

\paragraph{Arguments}

\begin{procarg}{--fibonacci-index=$m$ | --m=$m$ | --fibonacci-index $m$ | --m $m$}
  The Fibonacci index $m$ of the Fibonacci number $F_m$ is required to satisfy $m > 2$. The cardinality of the resulting point set equals $F_m$.
\end{procarg}

\begin{procarg}{--compute-fibonacci-number | --cardinality}
  Computes the point set’s cardinality and feeds it to the output stream. Here, this cardinality equals the Fibonacci number $F_m$, where $m$ is the Fibonacci index.
\end{procarg}

\begin{procarg}{--no-pointset}
  Emit no point set. Useful when measuring cardinality only.
\end{procarg}

\procargout

\procargdelimiter

\procargsilent

\paragraph{Limitations}
Kritzinger's modified Fibonacci lattice is two-dimensional, only.
