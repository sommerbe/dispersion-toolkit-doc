%
A useful visual debugging, or inspection, tool is to visualise point set sequences, stored in the file $\codef{pts.seq}$. Here, any other filename would do it, for instance $\codef{pts-seq.csv}$ would be fine as well. This file includes the domain boundary for each point set therein, which becomes essential for this visualisation task since the $\sdim=2$ dimensional plotter needs to properly limit both axes in preventing false impressions.

Visualising such a sequence is made simple by having only to execute the command
\begin{align*}
  \codef{python ./bin/vis-pss.py --i=pts.seq}
\end{align*}
requiring matplotlib, numpy and python3 as dependencies. Notice, however, that the computational performance of this procedure is non-optimal. Particularly large point sets may require substantial time to be drawn.

For the sequence $\fnP_i$, $i \in [0,\ptseqsize)$, this command essentially generates an animation of $\ptseqsize$ frames on the fly. These frames could be stored permanently to
\begin{align*}
  \codef{f0.png}, \codef{f1.png}, ..., \codef{f$\ptseqsize-1$.png}
\end{align*}
following the pattern $\codef{f\{i\}.png}$. The option $\codef{--image-path=f{i}.png}$ generates these files. So the above command is extended with
\begin{align*}
  \codef{python ./bin/vis-pss.py --i=pts.seq --image-path=f\{i\}.png}
\end{align*}
for which the resolution of these images could optionally be specified with $\codef{--image-ppi=600}$.

\paragraph{Greatest empty rectangles}

The quest towards finding a point set with minimal dispersion may require to analyse its greatest empty rectangles, either those being local or global. This holds for dispersion involving rectangles, compared to other regions such as balls and triangles. The command
\begin{align*}
  \codef{./bin/disp-combinatorial --i=pts.csv --o=box.csv --boxes}
\end{align*}
computes all locally greatest empty rectangles, bounded by the problem domain specified in $\codef{pts.csv}$ storing a point set. These rectangles are stored permanently to $\codef{box.csv}$. With this preprocessing, 
\begin{align*}
  \codef{./bin/vis-rectangles.py --i=box.csv}
\end{align*}
draws these rectangles within the problem domain boundary.\footnote{Currently, the problem domain boundary of a rectangle set is not stored within this file, such as within $\codef{box.csv}$. Currently, this domain needs to be specified manually with $\codef{--domain}$ when executing $\codef{./bin/vis-rectangles.py}$, if not $[0,1]^2$.} Notice that this command accepts the option $\codef{--image-path}$ as well, besides quite a few other customisation options. Furthermore, not all locally greatest empty rectangles need to be streamed. For instance,
\begin{align*}
  \codef{./bin/disp-combinatorial }&\codef{--i=pts.csv --o=box.csv}\\
  &\codef{--greatest-box}
\end{align*}
emits the globally greatest rectangle only.

\paragraph{Finding area of greatest empty rectangle}

The area $a$ of the greatest empty rectangle is found by executing the command
\begin{align*}
  \codef{./bin/disp-combinatorial --i=pts.csv --greatest-box --box-area}
\end{align*}
which returns a $\sdim=5$ dimensional point set whereas the last axis\footnote{The last axis corresponds to the last number of each number tuple, stored in a distinct line.} relates to the area of the rectangle. For instance,
\begin{verbatim}
2e-01 3e-01 8e-01 7e-01 3.64e-01
#eos
\end{verbatim}
is an axis-aligned rectangle $[\num{2e-1},\num{8e-1}] \times [\num{3e-1},\num{7e-1}]$ with area $a=\num{3.64e-01}$.

\paragraph{Greatest empty rectangles with minimum area}

Having found a minimum rectangle area $a$, 
\begin{align*}
  \codef{./bin/disp-combinatorial }&\codef{--i=pts.csv --o=box.csv}\\
  &\codef{--boxes --box-area-min=$a$}
\end{align*}
streams only those rectangles having at least an area $\geq a$, for instance
\begin{align*}
  \codef{./bin/disp-combinatorial }&\codef{--i=pts.csv --o=box.csv}\\
  &\codef{--boxes --box-area-min=3.64e-1}
\end{align*}
stores all rectangles satisfying this condition to $\codef{box.csv}$. Again,
\begin{align*}
  \codef{./bin/vis-rectangles.py --i=box.csv}
\end{align*}
draws these rectangles.
