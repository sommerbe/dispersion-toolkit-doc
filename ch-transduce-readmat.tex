%
This algorithm reads an externally encoded matrix of $n$ points $\bm p \in \fnP_i$ into the internally employed format of this dispersion toolkit.

A $\sdim$ dimensional point $\bm p$, with coordinates $p_i$, $i \in [0,d)$, shall be externally encoded as
\begin{align*}
  F[\bm p] \coloneqq \fttoken{pt-prefix}p_0\fttoken{coord-delimiter}p_1\fttoken{coord-delimiter}...p_{d-1}\fttoken{pt-suffix}
\end{align*}
to be used repeatedly within the externally encoded matrix. This matrix shall be formatted as
\begin{align*}
  M[\bm p_i \in \fnP] \coloneqq \fttoken{set-prefix}F[\bm p_0]\fttoken{pt-del}F[\bm p_1]\fttoken{pt-del}...F[\bm p_{n-1}]\fttoken{set-suffix}
\end{align*}
by an external software generating point sets. In case this software streams a sequence of point sets, $\fnP_i$, $i \in [0,\ptseqsize)$, these $M[\fnP_i]$ are concatenated without additional characters,
\begin{align}\mlabel{eq:transduce-rmat-s}
  S[\fnP_i, \ptseqsize] \coloneqq M[\fnP_0]M[\fnP_1]...M[\fnP_{\ptseqsize-1}].
\end{align}

The internally, decoded, point set sequence requires a definition of their domain boundary
\begin{align*}
  \fdom \coloneqq [a_j, b_j]^{\sdim}_{j \in [0,\sdim)}, a_j \leq b_j.
\end{align*}

This algorithm reads $S[\fnP_i, \ptseqsize]$.

\paragraph{Procedure}

\begin{synopsis}
  \subsynopsis{read-matrix}{[--i=FILE] [--o=FILE]\newline
  --domain-boundary=$a_0\ b_0\ a_1\ b_1...\ a_{\sdim-1}\ b_{\sdim-1}$\newline
  [--point-prefix=$\fttoken{pt-prefix}$] [--point-suffix=$\fttoken{pt-suffix}$]\newline
  [--coord-delimiter=$\fttoken{coord-delimiter}$]\newline
  [--point-delimiter=$\fttoken{pt-del}$]\newline
  [--set-prefix=$\fttoken{set-prefix}$] [--set-suffix=$\fttoken{set-suffix}$]\newline
  [--mathematica] [--csv]\newline
  [--delimiter=$\fttoken{delimiter}$]\newline
  [--silent]}
  \subsynopsis{read-matrix}{[--i=FILE] [--o=FILE]\newline
  --domain-boundary-unity\newline
  [--point-prefix=$\fttoken{pt-prefix}$] [--point-suffix=$\fttoken{pt-suffix}$]\newline
  [--coord-delimiter=$\fttoken{coord-delimiter}$]\newline
  [--point-delimiter=$\fttoken{pt-del}$]\newline
  [--set-prefix=$\fttoken{set-prefix}$] [--set-suffix=$\fttoken{set-suffix}$]\newline
  [--mathematica] [--csv]\newline
  [--delimiter=$\fttoken{delimiter}$]\newline
  [--silent]}
\end{synopsis}

\paragraph{Arguments}

\procarginseq{\ptseqsize}

\procargout

\begin{procarg}{--domain-boundary-unity}
  Sets $\fdom = [0,1]^d$.
\end{procarg}

\begin{procarg}{--point-prefix=$\fttoken{pt-prefix}$}
  Sets $\fttoken{pt-prefix}$.
\end{procarg}

\begin{procarg}{--point-suffix=$\fttoken{pt-suffix}$}
  Sets $\fttoken{pt-suffix}$.
\end{procarg}

\begin{procarg}{--coord-delimiter=$\fttoken{coord-delimiter}$}
  Sets $\fttoken{coord-delimiter}$.
\end{procarg}

\begin{procarg}{--point-delimiter=$\fttoken{pt-del}$}
  Sets $\fttoken{pt-del}$.
\end{procarg}

\begin{procarg}{--set-prefix=$\fttoken{set-prefix}$}
  Sets $\fttoken{set-prefix}$.
\end{procarg}

\begin{procarg}{--set-suffix=$\fttoken{set-suffix}$}
  Sets $\fttoken{set-suffix}$.
\end{procarg}

\begin{procarg}{--mathematica}
  Sets 
  \begin{align*}
    \fttoken{pt-prefix} &\coloneqq \codef{\{} &
    \fttoken{coord-delimiter} &\coloneqq \codef{,} &
    \fttoken{pt-suffix} &\coloneqq \codef{\}} \\
    \fttoken{set-prefix} &\coloneqq \codef{\{} &
    \fttoken{pt-del} &\coloneqq \codef{,} &
    \fttoken{set-suffix} &\coloneqq \codef{\}}
  \end{align*}
  to be used within $S[\fnP_i, \ptseqsize]$.
\end{procarg}

\begin{procarg}{--csv}
  Sets 
  \begin{align*}
    \fttoken{pt-prefix} &\coloneqq \codef{} &
    \fttoken{coord-delimiter} &\coloneqq \codef{\textvisiblespace } &
    \fttoken{pt-suffix} &\coloneqq \codef{} \\
    \fttoken{set-prefix} &\coloneqq \codef{} &
    \fttoken{pt-del} &\coloneqq \formateol &
    \fttoken{set-suffix} &\coloneqq \codef{}
  \end{align*}
  to be used within $S[\fnP_i, \ptseqsize]$.
\end{procarg}

\begin{procarg}{--delimiter=$\fttoken{delimiter}$}
  Sets $\fttoken{delimiter}$ being the separating ASCII character between coordinates of a point $\bm p \in \fnP_i$.
\end{procarg}


\procargsilent

\paragraph{Return format}

This procedure returns a $\sdim$ dimensional point set sequence $\fnP_i$ to be used by procedures within this dispersion toolkit.
