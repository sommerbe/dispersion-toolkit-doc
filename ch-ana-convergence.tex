Estimates gradients of a sequence of values, along each dimension.

Supported gradient computation methods:

\begin{enumerate}
  \item Finite forward differences
\end{enumerate} 

In case of dealing with graphs, for which the first dimension represents the function’s arguments, computing the gradients is not done for this first axis.

\paragraph{Procedure}

\begin{synopsis}
  \subsynopsis{ana-convergence}{[--i=FILE] [--o=FILE]\newline
  [--absolute] [--ffd] [--silent]}
  \subsynopsis{ana-convergence}{--graph-layout [--i=FILE] [--o=FILE]\newline
  [--absolute] [--ffd] [--silent]}
\end{synopsis}

\paragraph{Arguments}

\begin{procarg}{--absolute}
  Ensures that the approximated gradients, computed by the specified method, are non-negative.
\end{procarg}

\begin{procarg}{--ffd}
  Computes finite forward differences on the sequence of values to approximate the gradients.
\end{procarg}

\begin{procarg}{--graph-layout}
  Specifies that the first axis, or dimension, of the receiving point set represents the arguments of a function mapping to all other values. So each $\sdim$ dimensional point of this set is interpreted that

\begin{itemize}
  \item the first coordinate is the argument to a function,
  \item the second coordinate is the value of the first graph, until
  \item the $\sdim$-th coordinate is the value of the ($\sdim$-1)th graph.
\end{itemize}

This option requires the point set $\fnP$ to have a least d=2 dimensions.
\end{procarg}

\begin{procarg}{--i=FILE | --i FILE}
  Retrieves a point set $\fnP$ of dimension $\sdim$ from \codef{FILE}. Its absence is substituted by stdin. The end of a point set, which equals the line \codef{\#eos}, starts the algorithm.
\end{procarg}

\procargout

\procargsilent

\paragraph{Return format}

The returned point set has equal dimension compared to the given input. But its cardinality is reduced, depending on the chosen method of how to compute the gradient  approximations.

In case of --graph-layout, values of the first axis remain constant.

