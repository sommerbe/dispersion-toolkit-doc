
The Fibonacci lattice\footnote{For more information, refer to \ct{kritzinger2020}.} based on the Fibonacci number $F_i$ was defined as
\begin{align*}
  F_i &\coloneqq \begin{cases}
    F_{i-2} + F_{i-1} & \text{if}\ i \in \mathbb N_{\geq 3}, \\
    1 & \text{if}\ i \in \{1,2\},
  \end{cases}\\
  \{r\} &\coloneqq r - \lfloor r \rfloor,\ \forall r \in \mathbb R_{> 0}, \\
  \fnPi{F}(m) &\coloneqq \mleft\{ \mleft(k F_m^{-1},\ \{k F_{m-2} F_m^{-1}\} \mright)\ .\ \forall k \in [0,F_m-1]_{\mathbb N_0} \mright\},
\end{align*}
while \cto[p.~70]{graham1994} defined the fractional part $\{r\}$.

\paragraph{Procedure}

\begin{synopsis}
  \subsynopsis{set-fibonacci}{(--fibonacci-index|--m)=$i$\newline 
    [--o=FILE]\newline
    [--delimiter=CHARACTER] [--silent]}
  \subsynopsis{set-fibonacci}{(--fibonacci-index|--m)=$i$\newline
    --compute-fibonacci-number|--cardinality [--no-pointset]\newline
    [--o=FILE]\newline
    [--delimiter=CHARACTER] [--silent]}
\end{synopsis}

\paragraph{Arguments}

\begin{procarg}{--fibonacci-index=$i$ | --m=$i$ | --fibonacci-index $i$ | --m $i$}
  The Fibonacci index $i$ of the Fibonacci number $F_i$ is required to satisfy $i > 2$. The cardinality of the resulting point set equals $F_i$.
\end{procarg}

\begin{procarg}{--compute-fibonacci-number | --cardinality}
  Computes the point set’s cardinality and feeds it to the output stream. Here, this cardinality equals the Fibonacci number $F_i$, where $i$ is the Fibonacci index.
\end{procarg}

\begin{procarg}{--no-pointset}
  Emit no point set. Useful when measuring cardinality only.
\end{procarg}

\procargout

\procargdelimiter

\procargsilent

\paragraph{Limitations}
The Fibonacci lattice will be two-dimensional, only.

