%
For $n$ points $\bm p_i \in \fnP_i$, the algorithm initially precomputes
\begin{align*}
  d(\bm p_i, \bm p_j)\ \forall \bm p_i, \bm p_j \in \fnP_i
\end{align*}
being stored in an indexed array.

Next, all $p$-element combinations of $\mathcal U$ are computed using a recursion. To this end, the algorithm determines the first index $i_0$ of $\bm p_{i_0} \in \mathcal U$ with the outer loop $i_0 = 0 \rightarrow n-1$. The subsequently nested loop $i_1 = i_0 \rightarrow n-1$ determines the second index $i_1$. This recursion continues until $i_{p-1}$. So,
\begin{align*}
  &i_0 = 0 \rightarrow n-1 \\
  &\quad i_1 = i_1+1 \rightarrow n-1 \\
  &\qquad\quad \vdots \\
  &\qquad\quad i_{p-1} = i_{p-2}+1 \rightarrow n-1.
\end{align*}
This scheme selects the $p$-element combination
\begin{align*}
  u \coloneqq (i_0, i_1, ..., i_{p-1}),\quad [i_a = i_b] = [a = b]\ \forall a,b
\end{align*}
where $[\cdot = \cdot]$ is the Iverson bracket mapping a boolean operation to binary integers. Therefore,
\begin{align*}
  \mathcal U = \{\bm p_a \}_{a \in u}.
%  \mathcal U = \{\bm p_{i_0}, \bm p_{i_1}, ..., \bm p_{i_{p-1}} \}.
\end{align*}

The minimal mutual distances are determined within two nested loops
\begin{align*}
  &a = 0 \rightarrow p-1 \\
  &\quad b = a+1 \rightarrow p-1 
\end{align*}
using
\begin{align*}
  \min_{a,b} d(\bm p_{u_a}, \bm p_{u_b}).
\end{align*}
Finally,
\begin{align*}
  \max_{u} \min_{a,b} d(\bm p_{u_a}, \bm p_{u_b})
\end{align*}
concludes the computation. The preprocessing requires a computational complexity of $\bigO(n^2)$ in time as well as in space. Determining $u$ causes the greatest complexity with $\bigO(n^p)$ in time, while $\min_{a,b}$ multiplicatively causes $\bigO(p^2)$. In total, this exhaustive search amounts to a computational complexity of $\bigO(n^p p^2)$ in time and of $\bigO(n^2)$ in space.

Fortunately, this algorithm works for any $d$ dimensional point set $\fnP_i$.

\paragraph{Procedure}

\begin{synopsis}
  \subsynopsis{pdisp-permute}{[--i=FILE] [--o=FILE]\newline
  --p=$p$\newline
  [--disp] [--ndisp] [--silent]}
\end{synopsis}

\paragraph{Arguments}

\procarginseq{\ptseqsize}

\procargout

\begin{procarg}{--disp}
  Computes $\pdisp(\fnP_i, p)$.
\end{procarg}

\begin{procarg}{--ndisp}
  Computes $n \pdisp(\fnP_i, p)$, with point set cardinality $n=\abs{\fnP_i}$.
\end{procarg}

\begin{procarg}{--p=$p$ | --p $p$}
  Sets the cardinality of combinations $\mathcal U$, or $u$ in the algorithmic description above. Boundary condition: $1 < p \leq n$.
\end{procarg}

\procargsilent

\paragraph{Return format}

The measurements
\begin{align*}
  t_i &= \mleft( \pdisp(\fnP_i, p),\quad \abs{\fnP_i} \pdisp(\fnP_i, p) \mright)
\end{align*}
represent a point of the streamed point set with cardinality $\ptseqsize$. Depending on the arguments given, some elements of $t_i$ may be omitted, but not all. Therefore, the stream 
\begin{align*}
  &t_0\formateol\\
  &t_1\formateol\\
  &\vdots \\
  &t_{\ptseqsize-1}\formateol\\
  &\formateos\formateol
\end{align*}
results.
