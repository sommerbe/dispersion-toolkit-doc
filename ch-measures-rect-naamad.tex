
\cto[algorithm MERAlg1]{naamad1984merp} essentially proposed a linear sweeping algorithm, both without hierarchies and without complex geometric reasoning. The idea is to having to consider only those locally greatest empty rectangles $\bm r \in \Rrect$ which could actually contribute to the globally greatest one. 

The algorithm sweeps in three steps. First, all exteriour empty rectangles are checked maximally spanning along $d=1$ within the domain boundary. This step requires a sorting of $n$ points $\bm p_i \in \fnP$ along $d=0$ to ensure the emptiness condition while keeping computational performance in mind.

Second, all candidate interiour empty rectangles are considered. To this end, the algorithm starts with an empty rectangle $\bm r$ whose extent is maximal along the $d=0$ axis domain boundary, and whose extent is zero along $d=1$, for each point $\bm p_i$. By iterating through all points $\bm p_j$ with $p_{j,1} < p_{j,1}$, their coordinates $p_{j,0}$ along $d=0$ shrink $\bm r$, while $\bm r$ grows along $d=1$ to be bounded by $p_{j,1}$ from below. 

Third, all exteriour empty rectangles matching the upper domain boundary along $d=1$ are considered. These are bounded by $p_{i,1}$ from below, along $d=1$. Computationally, the second step leads to $\bigO(n^2)$ in time and $\bigO(n)$ in memory. 

This dispersion measure follows (\mref{eq:disp-range}) with
\begin{align*}
  \fndisp(\fnP_i, \Rrect, \muarea)
\end{align*}
while being limited to $\fnP_i \subset \setR^2$.

\paragraph{Procedure}

\begin{synopsis}
  \subsynopsis{disp-naamad}{[--i=FILE] [--o=FILE]\newline
  [--disp] [--ndisp] [--count-boxes]\newline
  [--silent]}
\end{synopsis}

\paragraph{Arguments}

\begin{procarg}{--disp}
  Computes $\fndisp(\fnP_i, \Rrect, \muarea)$.
\end{procarg}

\begin{procarg}{--ndisp}
  Computes $\abs{\fnP_i} \fndisp(\fnP_i, \Rrect, \muarea)$, with point set cardinality $\abs{\fnP_i}$.
\end{procarg}

\begin{procarg}{--count-boxes}
  Counts all empty \enquote{restricted rectangles} (RRs) of $\fnP_i$, considered as RRs in the publication \cp{naamad1984merp}.
\end{procarg}

\procarginseq{\ptseqsize}

\procargout

\procargsilent

\paragraph{Return format}

This procedure returns a point set of cardinality $n$ with dimension $0 < d \leq 3$ while each axis represents a computed measure,

\begin{tabular}{cccc}
  $\fnP_i$ & \codef{--disp} & \codef{--ndisp} & \codef{--count-boxes} \\
  \toprule
  $0$ & \cdot & \cdot & \cdot \\
  $1$ & \cdot & \cdot & \cdot \\
  $\vdots$ & $\vdots$ & $\vdots$ & $\vdots$ \\
  $\ptseqsize-1$ & \cdot & \cdot & \cdot, \\
  \bottomrule  
\end{tabular}

being in this column-wise order, and while some columns are optional but not all. 


