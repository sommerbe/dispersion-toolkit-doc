To begin with, the source of this toolkit is obtained by cloning the git repository
\begin{align*}
  \codef{git@github.com:sommerbe/dispersion-toolkit.git}
\end{align*}
using the command
\begin{align*}
  \codef{git clone git@github.com:sommerbe/dispersion-toolkit.git}
\end{align*}
which works in Linux Bash, Windows (cmd, PowerShell) as well as in MacOS. Within the cloned repository,
\codeline{cd ./dispersion-toolkit},
create an out of source build directory
\codeline{mkdir build}
and within it using
\codeline{cd ./build}
configure the build using CMake's
\codeline{cmake ..}.
CMake is obtained from \href{https://cmake.org/}{https://cmake.org/}.


\paragraph{Linux Operating Systems: make}

In Linux operating systems, the entire toolkit is built with
\codeline{make -j$T$}
where $T$ is the number of parallel builds, usually being the number of actual parallels threads the CPU supports, for instance $T=8$. Upon successful completion, the sub directory
\codeline{cd ./bin}
contains the procedures shipped by this toolkit. Being within this directory is henceforth assumed when working directly from this build directory, i.e. when installs are omitted.


\paragraph{Windows Operating Systems: Visual Studio}

The build directory contains the solution
\codeline{dispersion-toolkit.sln}
to be opened with Visual Studio. Executing a Debug build generates the procedures within
\codeline{./bin/Debug}
and within
\codeline{./bin/Release}
in case of release builds. Within
\codeline{./bin}
are located Python scripts. This is different to the make build system which stores both scripts and compiled binaries into \codef{./bin} where the build directory \codef{../build} is for instance debug or release mode.

Running the compiled procedures henceforth assumes to, for instance,
\codeline{cd ./bin/Debug}
to enter the directory containing the executables in case of omitting installs.

