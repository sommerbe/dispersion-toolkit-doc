\newcommand{\procarginseq}[1]{%
\begin{procarg}{--i=FILE | --i FILE}
  Retrieves a point set sequence $\fnP_i$ of size $#1$ from \codef{FILE}, with $i \in [0,#1)$. Its absence is substituted by stdin. The end of a point set, which equals the line \codef{\#eos}, starts the algorithm.
\end{procarg}}

\newcommand{\procargout}{%
\begin{procarg}{--o=FILE | --o FILE}
  Redirects the computed results to \codef{FILE}, opened in overwrite mode (not appending mode). Without \codef{FILE}, results are forwarded to stdout. Errors encountered during the  program’s execution are streamed into stderr, and not into either stdout or \codef{FILE}.
\end{procarg}}

\newcommand{\procargdelimiter}{%
\begin{procarg}{--delimiter=CHARACTER}
  Define a delimiter character to separate coordinates of a point in the resulting output. A usual \codef{CHARACTER} could be a whitespace (\codef{' '}) or tabular character (\codef{'\chartab'}), for instance.
\end{procarg}}

\newcommand{\procargsilent}{%
\begin{procarg}{--silent}
  Suppress comments in the output stream, yielding only the computed value. The latter could be the point set or its cardinality.
\end{procarg}}
