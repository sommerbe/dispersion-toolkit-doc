\documentclass[12pt,a4paper,refnotes]{extbook}


% typesetting
\usepackage{amsmath}     % advanced math
\usepackage{mathtools}
\usepackage{bm}
\usepackage{amssymb}     % math symbols (square)
%\usepackage{dsfont}      % for mathds (identity matrix symbol)
\usepackage{commath}     % correct math differential typesetting (ISO 31/X)
\usepackage{siunitx}     % SI units formatting
\sisetup{output-exponent-marker=\ensuremath{\mathrm{e}}}
%\sisetup{output-exponent-marker=}

\usepackage{microtype}
\usepackage{mleftright}

% linking
\usepackage{hyperref}
\hypersetup{
  bookmarks=true,
  unicode=true,
  pdftitle={Dispersion toolkit manual},
  pdfauthor={The contributors},
  pdfsubject={documentation},
  pdfcreator={LaTeX},
  pdfkeywords={dispersion, manual, documentation},
  colorlinks=true,
  linkcolor=black,
  citecolor=black,
  filecolor=black,
  urlcolor=black,
  linktoc=page
}

% language
\usepackage[english]{babel}
\usepackage{csquotes}

% paragraph
\usepackage{titlesec}
\titleformat{\paragraph}
    {\normalfont\bfseries}
    {}
    {0pt}
    {}
\usepackage{parskip}
    
% bibliography
\usepackage[backend=biber
  ,style=authoryear
  ,sorting=anyt
  ,maxalphanames=1
  ,block=none]{biblatex}
\renewcommand*{\bibfont}{\footnotesize}

% figures and subfigures (package subcaption)
\usepackage{graphicx}
\usepackage{subcaption}
\graphicspath{{figures/}}

% picture drawing package
\usepackage{tikz}
\usetikzlibrary{datavisualization}

% tabulars
\usepackage{booktabs}
\setlength{\tabcolsep}{12pt}

% define matters
% thanks to: https://tex.stackexchange.com/questions/345263/latex-article-frontmatter
\def\frontmatter{%
    \pagenumbering{roman}
    \setcounter{page}{1}
    \renewcommand{\thesection}{\Roman{section}}
}%
\def\mainmatter{%
    \pagenumbering{arabic}
    \setcounter{page}{1}
    \setcounter{section}{0}
    \renewcommand{\thesection}{\arabic{section}}
}%
\def\backmatter{%
    \setcounter{section}{0}
    \renewcommand{\thesection}{\Alph{section}}
}%

% section inclusion
\newcommand{\chapterf}[2]{\chapter{#1}\input{#2}}
\newcommand{\sectionf}[2]{\section{#1}\input{#2}}
\newcommand{\subsectionf}[2]{\subsection{#1}\input{#2}}
\newcommand{\subsubsectionf}[2]{\subsubsection{#1}\input{#2}}

% references (intern, extern): helping tools
\usepackage{refnote}

% special characters
\newcommand{\chartab}{\symbol{'134}t}


% procedure argument typeface
\newcommand{\codef}[1]{\texttt{#1}}

% procedure synopsis
\newenvironment{synopsis}%
{\begin{description}}
{\end{description}}

\newcommand{\subsynopsis}[2]{\item[\codef{#1}] \codef{#2}}

% procedure argument
\newenvironment{procarg}[1]%
{\codef{#1}\newline}{}

% code enviornment
\newcommand{\codeline}[1]{\begin{align*}
  \codef{#1}
\end{align*}}

% for debugging purposes
% \usepackage{showframe}


% standard math operators
\DeclareMathOperator*{\argmax}{arg\!\max}
\DeclareMathOperator*{\argmin}{arg\!\min}
\DeclareMathOperator*{\arginf}{arg\!\inf}
\DeclareMathOperator*{\argsup}{arg\!\sup}
\DeclareMathOperator*{\fndisp}{disp}
\DeclareMathOperator*{\disp}{disp}
\DeclareMathOperator*{\pdisp}{p-disp}

% dimension
\newcommand{\sdim}{d}

% point set
\newcommand{\fnPi}[1]{\mathcal P_{\text{#1}}}
\newcommand{\fnP}{\mathcal P}

% optimised point set
\newcommand{\fnoP}{\widehat{\mathcal P}}

% region shape (rectangular, box, torus)
\newcommand{\fnR}{\mathcal R}
\newcommand{\fnRi}[1]{\mathcal R_{\text{#1}}}
\newcommand{\Rrect}{\fnRi{rect}}
\newcommand{\Rtorus}{\fnRi{torus}}

% measure of region shape
\newcommand{\fnMui}[1]{\mathcal \mu_{\text{#1}}}
\newcommand{\muarea}{\fnMui{area}}
\newcommand{\muvol}{\fnMui{vol}}

% point set sequence size
\newcommand{\ptseqsize}{m}

% computational complexity classes
\newcommand{\ccNPhard}{$\mathcal{NP}$-hard}
\newcommand{\ccNPcomplete}{$\mathcal{NP}$-complete}

% mathematical standard sets
\newcommand{\setR}{\mathbb{R}}
\newcommand{\setN}{\mathbb{N}}

% asymptotic computational bounds
\newcommand{\bigO}{\mathcal O}
\newcommand{\bigW}{\Omega}%\mathcal W}

% arithmetic manipulation: floor, ceiling
\newcommand{\floor}[1]{\lfloor#1\rfloor}
\newcommand{\ceil}[1]{\lceil#1\rceil}



\hyphenation{}


\newcommand{\procargout}{%
\begin{procarg}{--o FILE | --o=FILE}
  Redirects the computed results to \codef{FILE}, opened in overwrite mode (not appending mode). Without \codef{FILE}, results are forwarded to stdout. Errors encountered during the  program’s execution are streamed into stderr, and not into either stdout or \codef{FILE}.
\end{procarg}}

\newcommand{\procargdelimiter}{%
\begin{procarg}{--delimiter=CHARACTER}
  Define a delimiter character to separate coordinates of a point in the resulting output. A usual \codef{CHARACTER} could be empty space (\codef{' '}) or tabular character (\codef{'\chartab'}), for instance.
\end{procarg}}

\newcommand{\procargsilent}{%
\begin{procarg}{--silent}
  Suppress comments in the output stream, yielding only the computed value. The latter could be the point set or its cardinality.
\end{procarg}}

\addbibresource{bibliography.bib}

%\makeatletter
\title{Dispersion toolkit documentation}
\author{The contributors}
\date{24. February 2021}

\begin{document}
\frontmatter
\maketitle
\section*{Acknowledgements}
I would like to thank everyone who contributed to this project, by implementing computational capabilities or by providing constructive feedback for further development through both testing and using it.

Thank you,\\
Benjamin Sommer

\tableofcontents
\clearpage

\mainmatter
\chapterf{Introduction}{ch-intro}

\chapterf{Point sets and sequences}{ch-sets}
\sectionf{Fibonacci sets}{ch-sets-fibo}
\sectionf{Kritzinger's modified Fibonacci sets}{ch-sets-kritz}
\sectionf{Coordinate swapping}{ch-sets-cswap}


In \citeyear{hlawka1976disp}, \citeauthor{hlawka1976disp} described \enquote{Dichtigkeitsdiskrepanz} 
\begin{align*}
  \Delta(\omega) &\coloneqq \sup_{\varphi} \min_{j} \min_{g} L(\varphi, \theta_j, g),\ \text{with}\\
  L(\varphi, \theta, g) &\coloneqq \abs{\varphi - \theta - g},\\
  \omega &\coloneqq (\theta_1, \theta_2, ..., \theta_N),\quad g \in \mathbb Z,
\end{align*}
a problem seeking to find the greatest distance from $\varphi$ to $\theta_j$ and $g$ without having other numbers of the discrete $\theta_j$ and $g$ inbetween \cp{hlawka1976disp}. In other words, it could be viewed as an example of a greatest empty discrete interval problem.\footnote{This terminology shall be defined here, though it is hardly new due to its generality.}

With the submission of the PhD dissertation, \cto[section 6.1.8]{shamos1978compgeom} defined the largest empty circle problem. Being mathematically equivalent except for $\sup$ instead of $\max$, \ct{niederreiter1979disp}\footnote{\cto[Definition]{niederreiter1983disp} defined this concept of dispersion again without referencing his previous publication in \citeyear{niederreiter1979disp}.} independently generalised the concept of \citeauthor{hlawka1976disp} with the introduction of \enquote{dispersion} in $d$-dimensions,\footnote{The notational symbols have been slightly changed without changing the mathematical meaning.}
\begin{align*}
  \fndisp(\mathcal E, n) &\coloneqq \sup_{\bm x \in \mathcal E} \min_{1 \leq i \leq n} d(\bm x, \bm x_{i}),\ \text{with}\\
  &\bm x_1, \bm x_2, ..., \bm x_{n} \in \mathcal E\ \wedge\ \bm x_i \in \mathbb R^d,
\end{align*}
to compute the greatest empty distance between given points $\bm x_i$. 

This dispersion measure need not to be limited to 1-simplices. For instance, \ct{rote1996disp} generalised this concept to 
\begin{align*}
  \text{ranges $\fnR$ equipped with a Lebesgue measure $\mu$.}
\end{align*}
A range $\fnR$ denotes a shape described with a fixed rule to reflect the size of a hole in the point set $\mathcal E$. The dispersion
\begin{align}\mlabel{eq:disp-range}
  \fndisp(\mathcal E, \fnR, \mu) \coloneqq \sup_{\bar R\,\in\,\mathcal R,\, \bar R\,\cap\,\mathcal E = \emptyset} \mu(\bar R)
\end{align}
measures the greatest range $\bar R$ out of all possible $\fnR$.

This chapter introduces procedures measuring (\mref{eq:disp-range}) for particular shapes of $\fnR$ and dimension $d$. For equal $\fnR$ and $d$, this toolkit provides implementations of different algorithms contributed throughout academic history to establish to common base. A base which possibly improves both reproducibility and comparison of novel algorithms regarding computational performance. In following this goal, applying dispersion to solving problems, both practical and theoretical, might be expected to spread.

\clearpage




\chapter{Statistics}

\chapter{Point set optimisation}

\chapter{Visualisation}

\chapter{Transduction}

%
Special thanks goes to:
\begin{description}
  \item[Thomas Lachmann] for numerous suggestions and testing.
\end{description}


\clearpage

\backmatter
\appendix
\printbibliography[heading=bibintoc]

\end{document}
