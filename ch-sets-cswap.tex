
Of any point set with dimension
\begin{align*}
  d > 0,
\end{align*}
swapping coordinates of these points helps in analysing the behaviour of the set with respect to some measures. 

The algorithm stochastically chooses 
\begin{align*}
  n\ \text{pairs of points}
\end{align*}
being indenpendently and uniformly identically distributed. An alternative Mersenne Twister, implemented as \codef{std::mt19937\_64} in the \symcpponeone{} standard generates the pseudo-random numbers employed for this purpose.

Instead of explicitly stating $n$, the algorithm may choose $n$ as a function of both the point set's cardinality $\abs{\fnP}$ and a percentage $p$, using
\begin{align}\label{eq:n-cswap}
  n = n_{\text{cswap}}\mleft(\fnP, p\mright) \coloneqq \floor{\,p\abs{\fnP}},\quad p \in [0,1]_{\setR}.
\end{align}
In the absense of both $n$ and $p$,
\begin{align*}
  n=1.
\end{align*}

The axis,
\begin{align*}
  a \in \{0,1,...,d-1\}_{\setN},
\end{align*}
along which to swap is, by default, pseudo-randomly chosen. As an alternative, this additional randomisation may become deterministic by explicitly directing an axis index.

The stochastic nature of this algorithm enables the streaming of point sets, leading to a sequence of
\begin{align*}
  r\ \text{point sets}, r > 0, r \in \setN.
\end{align*}
If a dispersion measuring procedure is applied to this stream, and subsequently applying a statistical descriptor, a statistics of this randomisation with respect to the chosen dispersion measure arises. This allows to investigate the initial point set from a stochastic perspective.

\paragraph{Procedure}

\begin{synopsis}
  \subsynopsis{set-cswap}{[--i=FILE] [--o=FILE]\newline
  [(--count|--c)=$n$] [(--percentage|--p)=$p$]\newline
  [--axis=($-1$|$a$)]\newline
  [(--repeat|--r)=$r$]\newline
  [--silent]}
\end{synopsis}


\paragraph{Arguments}

\begin{procarg}{--count=$n$ | --c=$n$ | --count $n$ | –c $n$}
  The number of pairs of points for which to swap their coordinates (pairwise). By default, $n$=1.
\end{procarg}

\begin{procarg}{--percentage=$p$ | --p=$p$ | --percentage $p$ | –p $p$}
  The number of pairs of points, determined dynamically by evaluating (\mref{eq:n-cswap}).
\end{procarg}

\begin{procarg}{--axis=$-1$ | --axis=$a$ | --axis $a$}
  The axis along which to swap the coordinates. Randomisation with respect to the axis is enabled with $a=-1$.
\end{procarg}

\begin{procarg}{--repeat=$r$ | --r=$r$ | --repeat $r$ | --r $r$}
  The size of the resulting point set sequence.
\end{procarg}

\procarginseq{1}

\procargout

\procargdelimiter

\procargsilent


