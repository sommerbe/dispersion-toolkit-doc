%
This utility visualises a point set sequence $\fnP_i$ in $\sdim=2$ dimensions. The origin of this sequence could be either a point set generator itself, some randomisation of it, or intermediate results obtained during some optimisation procedure, as described in previous chapters.

In particular, this procedure focuses on how $\fnP_i$ change with $i$.

\paragraph{Procedure}

\begin{synopsis}
  \subsynopsis{vis-psspy.py}{[--i=FILE]\newline
  [--delay=$t$]\newline
  [--image-path=$s$] [--image-ppi=$r_{\text{ppi}}$]\newline
  [--silent]}
\end{synopsis}

\paragraph{Arguments}

\procarginseq{\ptseqsize}

\procargout

\begin{procarg}{--delay=$t$}
  Inserts a temporal delay of $t > 0$ seconds inbetween frames $f_i$ where each $f_i$ corresponds to the drawing $\fnP_i$.
\end{procarg}

\begin{procarg}{--image-path=$s$}
  Enables to stream the plotted frames $f_i$ to primary storage with a file name to match the pattern $s$. Here, $s$ is a principal file name containing the character sequence \codef{\{i\}} to be replaced by $i$ of $f_i$. For instance, \codef{seq-\{i\}.png} results to \codef{seq-0.png}, \codef{seq-1.png}, ..., \codef{seq-$\ptseqsize$.png}.
\end{procarg}

\begin{procarg}{--image-ppi=$r_{\text{ppi}}$}
  Sets the image resolution $r_{\text{ppi}}$ of the images to be stored, given in the pixel per inch unit, being the default unit of the underlying library used to generating these plots.
\end{procarg}

\procargsilent


\paragraph{Return format}

Nothing useful for further processing, except some output for informational purposes. The optionally generated images are streamed directly to primary storage.
